% Options for packages loaded elsewhere
\PassOptionsToPackage{unicode}{hyperref}
\PassOptionsToPackage{hyphens}{url}
\documentclass[
]{article}
\usepackage{xcolor}
\usepackage[margin=1in]{geometry}
\usepackage{amsmath,amssymb}
\setcounter{secnumdepth}{-\maxdimen} % remove section numbering
\usepackage{iftex}
\ifPDFTeX
  \usepackage[T1]{fontenc}
  \usepackage[utf8]{inputenc}
  \usepackage{textcomp} % provide euro and other symbols
\else % if luatex or xetex
  \usepackage{unicode-math} % this also loads fontspec
  \defaultfontfeatures{Scale=MatchLowercase}
  \defaultfontfeatures[\rmfamily]{Ligatures=TeX,Scale=1}
\fi
\usepackage{lmodern}
\ifPDFTeX\else
  % xetex/luatex font selection
\fi
% Use upquote if available, for straight quotes in verbatim environments
\IfFileExists{upquote.sty}{\usepackage{upquote}}{}
\IfFileExists{microtype.sty}{% use microtype if available
  \usepackage[]{microtype}
  \UseMicrotypeSet[protrusion]{basicmath} % disable protrusion for tt fonts
}{}
\makeatletter
\@ifundefined{KOMAClassName}{% if non-KOMA class
  \IfFileExists{parskip.sty}{%
    \usepackage{parskip}
  }{% else
    \setlength{\parindent}{0pt}
    \setlength{\parskip}{6pt plus 2pt minus 1pt}}
}{% if KOMA class
  \KOMAoptions{parskip=half}}
\makeatother
\usepackage{color}
\usepackage{fancyvrb}
\newcommand{\VerbBar}{|}
\newcommand{\VERB}{\Verb[commandchars=\\\{\}]}
\DefineVerbatimEnvironment{Highlighting}{Verbatim}{commandchars=\\\{\}}
% Add ',fontsize=\small' for more characters per line
\usepackage{framed}
\definecolor{shadecolor}{RGB}{248,248,248}
\newenvironment{Shaded}{\begin{snugshade}}{\end{snugshade}}
\newcommand{\AlertTok}[1]{\textcolor[rgb]{0.94,0.16,0.16}{#1}}
\newcommand{\AnnotationTok}[1]{\textcolor[rgb]{0.56,0.35,0.01}{\textbf{\textit{#1}}}}
\newcommand{\AttributeTok}[1]{\textcolor[rgb]{0.13,0.29,0.53}{#1}}
\newcommand{\BaseNTok}[1]{\textcolor[rgb]{0.00,0.00,0.81}{#1}}
\newcommand{\BuiltInTok}[1]{#1}
\newcommand{\CharTok}[1]{\textcolor[rgb]{0.31,0.60,0.02}{#1}}
\newcommand{\CommentTok}[1]{\textcolor[rgb]{0.56,0.35,0.01}{\textit{#1}}}
\newcommand{\CommentVarTok}[1]{\textcolor[rgb]{0.56,0.35,0.01}{\textbf{\textit{#1}}}}
\newcommand{\ConstantTok}[1]{\textcolor[rgb]{0.56,0.35,0.01}{#1}}
\newcommand{\ControlFlowTok}[1]{\textcolor[rgb]{0.13,0.29,0.53}{\textbf{#1}}}
\newcommand{\DataTypeTok}[1]{\textcolor[rgb]{0.13,0.29,0.53}{#1}}
\newcommand{\DecValTok}[1]{\textcolor[rgb]{0.00,0.00,0.81}{#1}}
\newcommand{\DocumentationTok}[1]{\textcolor[rgb]{0.56,0.35,0.01}{\textbf{\textit{#1}}}}
\newcommand{\ErrorTok}[1]{\textcolor[rgb]{0.64,0.00,0.00}{\textbf{#1}}}
\newcommand{\ExtensionTok}[1]{#1}
\newcommand{\FloatTok}[1]{\textcolor[rgb]{0.00,0.00,0.81}{#1}}
\newcommand{\FunctionTok}[1]{\textcolor[rgb]{0.13,0.29,0.53}{\textbf{#1}}}
\newcommand{\ImportTok}[1]{#1}
\newcommand{\InformationTok}[1]{\textcolor[rgb]{0.56,0.35,0.01}{\textbf{\textit{#1}}}}
\newcommand{\KeywordTok}[1]{\textcolor[rgb]{0.13,0.29,0.53}{\textbf{#1}}}
\newcommand{\NormalTok}[1]{#1}
\newcommand{\OperatorTok}[1]{\textcolor[rgb]{0.81,0.36,0.00}{\textbf{#1}}}
\newcommand{\OtherTok}[1]{\textcolor[rgb]{0.56,0.35,0.01}{#1}}
\newcommand{\PreprocessorTok}[1]{\textcolor[rgb]{0.56,0.35,0.01}{\textit{#1}}}
\newcommand{\RegionMarkerTok}[1]{#1}
\newcommand{\SpecialCharTok}[1]{\textcolor[rgb]{0.81,0.36,0.00}{\textbf{#1}}}
\newcommand{\SpecialStringTok}[1]{\textcolor[rgb]{0.31,0.60,0.02}{#1}}
\newcommand{\StringTok}[1]{\textcolor[rgb]{0.31,0.60,0.02}{#1}}
\newcommand{\VariableTok}[1]{\textcolor[rgb]{0.00,0.00,0.00}{#1}}
\newcommand{\VerbatimStringTok}[1]{\textcolor[rgb]{0.31,0.60,0.02}{#1}}
\newcommand{\WarningTok}[1]{\textcolor[rgb]{0.56,0.35,0.01}{\textbf{\textit{#1}}}}
\usepackage{graphicx}
\makeatletter
\newsavebox\pandoc@box
\newcommand*\pandocbounded[1]{% scales image to fit in text height/width
  \sbox\pandoc@box{#1}%
  \Gscale@div\@tempa{\textheight}{\dimexpr\ht\pandoc@box+\dp\pandoc@box\relax}%
  \Gscale@div\@tempb{\linewidth}{\wd\pandoc@box}%
  \ifdim\@tempb\p@<\@tempa\p@\let\@tempa\@tempb\fi% select the smaller of both
  \ifdim\@tempa\p@<\p@\scalebox{\@tempa}{\usebox\pandoc@box}%
  \else\usebox{\pandoc@box}%
  \fi%
}
% Set default figure placement to htbp
\def\fps@figure{htbp}
\makeatother
\setlength{\emergencystretch}{3em} % prevent overfull lines
\providecommand{\tightlist}{%
  \setlength{\itemsep}{0pt}\setlength{\parskip}{0pt}}
\usepackage{bookmark}
\IfFileExists{xurl.sty}{\usepackage{xurl}}{} % add URL line breaks if available
\urlstyle{same}
\hypersetup{
  pdftitle={Introductie tot R},
  hidelinks,
  pdfcreator={LaTeX via pandoc}}

\title{Introductie tot R}
\author{}
\date{\vspace{-2.5em}}

\begin{document}
\maketitle

\section{Introductie}\label{introductie}

R is een programmeertaal en omgeving die veel wordt gebruikt voor
statistische analyse, datawetenschap en grafische weergave. Het is
open-source en biedt een breed scala aan pakketten voor verschillende
toepassingen. R is populair onder statistici, data-analisten en
onderzoekers vanwege zijn flexibiliteit en kracht.

Misschien denk je nu: ``Programmeren? Dat klinkt eng en moeilijk!'' Maak
je geen zorgen - dat is heel normaal. Veel mensen voelen angst wanneer
ze ``programmeertaal'' horen.

De realiteit is dat \textbf{R leren vergelijkbaar is met het verwerven
van een nieuwe taal}. Net zoals bij het leren van een vreemde taal, moet
je de fundamentele structuur begrijpen: functies (vergelijkbaar met
werkwoorden), operatoren (vergelijkbaar met voorzetsels), en logische
verbindingen (vergelijkbaar met voegwoorden).

Het verschil is dat R een formele syntaxis heeft die consistenter is dan
natuurlijke talen. Waar Nederlandse grammatica vol uitzonderingen zit,
volgt R-code logische patronen die je kunt leren en toepassen.

\textbf{De pedagogische aanpak van dit cursusmateriaal} is gebaseerd op
progressieve complexiteit en praktische toepassing. Het primaire doel is
studenten een solide fundament te bieden in R-programmering door
concepten systematisch op te bouwen, van basisfuncties naar geavanceerde
data-analysemethoden. Door middel van concrete voorbeelden en hands-on
oefeningen wordt de theoretische kennis direct gekoppeld aan praktische
vaardigheden.

Deze cursus vereist geen voorafgaande programmeerervaring of diepgaande
wiskundige achtergrond. Met consistente studie en praktijkoefening
kunnen studenten zelfstandigheid ontwikkelen in statistische
data-analyse en onderzoeksmethodologie.

\section{Voordelen van R}\label{voordelen-van-r}

R biedt verschillende significante voordelen voor academische en
professionele data-analyse:

\textbf{Open Source en Kosteloos:} R is volledig gratis beschikbaar en
wordt ontwikkeld door een wereldwijde gemeenschap van statistici en
programmeurs. Dit betekent dat studenten en onderzoekers toegang hebben
tot geavanceerde statistische tools zonder licentiekosten.

\textbf{Reproduceerbaarheid:} R-scripts maken onderzoek volledig
reproduceerbaar. Anderen kunnen exact dezelfde analyses uitvoeren met
dezelfde resultaten, wat essentieel is voor wetenschappelijke
integriteit en peer review. Reproduceerbaarheid is een kernprincipe van
open science. Door R-scripts bij te voegen als bijlage bij scripties of
te delen met uitgevers, kunnen onderzoekers en reviewers precies zien
welke stappen zijn ondernomen in de data-analyse. Deze transparantie
versterkt de geloofwaardigheid van het onderzoek en maakt onafhendelijke
verificatie mogelijk.

\textbf{Uitgebreide Online Ondersteuning:} De R-gemeenschap is zeer
actief, met uitgebreide documentatie, tutorials, forums (zoals Stack
Overflow), en online cursussen. Problemen worden snel opgelost door de
gemeenschap.

\textbf{R Markdown voor Academisch Schrijven:} Met R Markdown kunnen
onderzoekers hun analyses, tabellen, grafieken en tekst in één document
integreren. Dit elimineert de noodzaak om handmatig tabellen en figuren
te kopiëren naar Word-documenten, waardoor fouten worden vermeden en
tijd wordt bespaard bij het schrijven van scripties en
onderzoeksartikelen Bovendien genereert R Markdown automatisch
referenties en bibliografieën, en kan output worden geëxporteerd naar
verschillende formaten zoals Word (docx) of PDF, afhankelijk van de
vereisten van het tijdschrift of de instelling. Een belangrijk voordeel
is dat handmatige invoer van statistische waarden in tekst volledig
wordt vermeden - wanneer data of analyses worden bijgewerkt, veranderen
de waarden in de tekst automatisch mee, wat de consistentie en
nauwkeurigheid van het document garandeert.

\textbf{Uitgebreide Pakketbibliotheek:} Met meer dan 18.000 pakketten op
CRAN dekt R vrijwel elk statistisch en analytisch domein, van basis
statistiek tot machine learning en bioinformatica.

\textbf{Krachtige Visualisatiemogelijkheden:} R's grafische
mogelijkheden, vooral via ggplot2, stellen gebruikers in staat
professionele, publicatie-kwaliteit visualisaties te creëren.

\section{Installeren van R}\label{installeren-van-r}

R is beschikbaar voor Windows, macOS en Linux. Volg deze stappen om R te
installeren: 1. Bezoek de CRAN-website: Ga naar
\url{https://cran.r-project.org/}. 2. Kies je besturingssysteem: Klik op
de link voor jouw besturingssysteem (Windows, macOS, of Linux). 3.
Download R: Volg de instructies op de pagina om de nieuwste versie van R
te downloaden en te installeren.

Zie Figuur 1.1 van de CRAN-website en kies de juiste versie voor jouw
besturingssysteem.

\pandocbounded{\includegraphics[keepaspectratio,alt={R CRAN pagina}]{screenshots and images/R CRAN page.png}}
\textbf{Figuur 1.1:} R CRAN pagina

\section{Installeren van RStudio}\label{installeren-van-rstudio}

RStudio is een geïntegreerde ontwikkelomgeving (IDE) voor R die het
schrijven van code, het beheren van projecten en het visualiseren van
resultaten vereenvoudigt. Volg deze stappen om RStudio te installeren:
1. Bezoek de RStudio-website: Ga naar
\url{https://posit.co/download/rstudio} 2. Kies de gratis versie: Klik
op de link voor RStudio Desktop Open Source License. 3. Download
RStudio: Selecteer de juiste installer voor jouw besturingssysteem
(Windows, macOS, of Linux) en volg de instructies om RStudio te
installeren. 4. Optioneel: Kies ``add desktop shortcut'' wanneer daarom
gevraagd wordt voor gemakkelijke toegang, of open RStudio later via het
Start-menu onder geïnstalleerde programma's.

\pandocbounded{\includegraphics[keepaspectratio,alt={RStudio download pagina}]{screenshots and images/Install R studio.png}}
\textbf{Figuur 1.2:} RStudio download pagina

\section{RStudio Interface}\label{rstudio-interface}

Wanneer je RStudio voor het eerst opent, zie je een interface die is
opgedeeld in vier hoofdpanelen. Deze lay-out is ontworpen om efficiënt
werken met R mogelijk te maken.

\textbf{Source (Broncode):} Dit is het linkerbovenvenster waar je
R-scripts schrijft en bewerkt. Hier typ je je code voordat je deze
uitvoert. Je kunt meerdere bestanden tegelijk openen in tabs.

\textbf{Console:} Het linkerondervenster toont de R-console waar
commando's direct kunnen worden uitgevoerd. Hier zie je ook de output
van je code en eventuele foutmeldingen. De console toont ook informatie
over de R-versie bij het opstarten.

\textbf{Environment (Omgeving):} Het rechterbovenvenster toont alle
variabelen, datasets, en objecten die momenteel in het geheugen zijn
geladen. Dit helpt je om bij te houden welke data beschikbaar is voor
analyse.

\textbf{Output:} Het rechterondervenster bevat verschillende tabs: -
\textbf{Files:} Bestandsbeheer en navigatie door mappen -
\textbf{Plots:} Visualisaties en grafieken die je hebt gemaakt -
\textbf{Packages:} Beheer van geïnstalleerde R-pakketten -
\textbf{Help:} Documentatie en hulp voor R-functies - \textbf{Viewer:}
Voor het bekijken van HTML-output en interactieve content

\textbf{Figuur 1.3:} RStudio Interface Layout
\pandocbounded{\includegraphics[keepaspectratio,alt={RStudio Interface}]{screenshots and images/R studio source and console.png}}

\section{Mapstructuur voor
Reproduceerbaarheid}\label{mapstructuur-voor-reproduceerbaarheid}

Een georganiseerde mapstructuur is essentieel voor reproduceerbaar
onderzoek. Het zorgt ervoor dat alle projectbestanden logisch geordend
zijn en gemakkelijk te vinden.

\subsection{Stap 1: Hoofdmap aanmaken}\label{stap-1-hoofdmap-aanmaken}

Voordat je begint met je onderzoek, maak eerst een hoofdmap aan op een
locatie naar keuze op je computer (bijvoorbeeld op je D of E schijf).
Geef deze map een duidelijke naam zoals ``Thesis'', ``Masterscriptie'',
``Onderzoek2024'', of de specifieke titel van je onderzoek.

\subsection{Stap 2: Submappen aanmaken}\label{stap-2-submappen-aanmaken}

Maak binnen deze hoofdmap de volgende submappen aan via Windows
Verkenner:

\begin{enumerate}
\def\labelenumi{\arabic{enumi}.}
\tightlist
\item
  Navigeer naar je hoofdmap
\item
  Rechtermuisknop \textgreater{} Nieuwe map
\item
  Maak de volgende mappen aan:
\end{enumerate}

\begin{itemize}
\tightlist
\item
  \textbf{Data:} Bevat alle oorspronkelijke datasets, bewerkte data, en
  databestanden die gebruikt worden in het onderzoek.
\item
  \textbf{Script:} Alle R-scripts die gebruikt worden voor data-analyse,
  data-cleaning, en statistische berekeningen.
\item
  \textbf{Literatures:} Gerelateerde literatuur, onderzoeksartikelen, en
  referentiemateriaal die relevant zijn voor het onderzoek.
\item
  \textbf{Drafts:} Verschillende versies van het manuscript, van eerste
  concepten tot definitieve versies.
\item
  \textbf{Output:} Gegenereerde tabellen, figuren, grafieken, en andere
  resultaten van de analyses.
\end{itemize}

\textbf{Georganiseerde Mapstructuur - Voorbeeld:}

\begin{verbatim}
Thesis/
├── Data/
├── Script/
├── Literatures/
├── Drafts/
└── Output/
\end{verbatim}

\subsection{Stap 3: Werkmap (Working Directory)
Instellen}\label{stap-3-werkmap-working-directory-instellen}

\textbf{Wat is een Working Directory?} De working directory is de map
waar R standaard naar zoekt om bestanden te lezen en op te slaan. Het is
belangrijk om deze correct in te stellen.

\textbf{Traditionele Methode - setwd():} Zonder R Projects moet je
handmatig de werkmap instellen met de \texttt{setwd()} functie:

\begin{Shaded}
\begin{Highlighting}[]
\CommentTok{\# Handmatig werkmap instellen}
\FunctionTok{setwd}\NormalTok{(}\StringTok{"D:/Thesis"}\NormalTok{)}

\CommentTok{\# Controleer huidige werkmap}
\FunctionTok{getwd}\NormalTok{()}

\CommentTok{\# Nu kun je bestanden laden met volledige paden}
\NormalTok{data }\OtherTok{\textless{}{-}} \FunctionTok{read.csv}\NormalTok{(}\StringTok{"D:/Thesis/Data/my\_data.csv"}\NormalTok{)}
\end{Highlighting}
\end{Shaded}

\textbf{Nadelen van setwd():} - Paden werken niet op andere computers
(jouw pad bestaat niet bij anderen) - Scripts zijn niet reproduceerbaar
zonder aanpassingen - Moeilijk om met meerdere projecten tegelijk te
werken - Moet bij elke R sessie opnieuw ingesteld worden

\textbf{Oplossing: R Projects (behandeld in volgende sectie)} R Projects
lossen al deze problemen automatisch op! Door een R Project aan te
maken, wordt de werkmap automatisch ingesteld op de projectmap, waardoor
je relatieve paden kunt gebruiken die overal werken.

Deze georganiseerde structuur zorgt ervoor dat: - Alle
projectgerelateerde bestanden op één centrale locatie staan -
Collaborateurs en supervisors gemakkelijk toegang hebben tot alle
materialen - Het project reproduceerbaar is voor andere onderzoekers -
Versiecontrole en backup eenvoudig te beheren zijn

\section{R Project Aanmaken in Thesis
Map}\label{r-project-aanmaken-in-thesis-map}

Nu je een georganiseerde mapstructuur hebt aangemaakt, is het tijd om
een R Project aan te maken. Een R Project lost alle problemen op die we
zagen met \texttt{setwd()} en maakt je werk veel professioneler en
reproduceerbaar.

\textbf{Waarom R Projects?} R Projects bieden de \textbf{optimale
oplossing} voor werkmap-beheer:

\begin{itemize}
\tightlist
\item
  \textbf{Automatische werkmap:} RStudio stelt automatisch de werkmap in
  naar de locatie van het \texttt{.Rproj} bestand
\item
  \textbf{Geen setwd() nodig:} De werkmap wordt automatisch correct
  ingesteld bij het openen van het project
\item
  \textbf{Relatieve paden:} Je kunt bestanden laden met
  \texttt{"Data/my\_data.csv"} in plaats van
  \texttt{"D:/Thesis/Data/my\_data.csv"}
\item
  \textbf{Portabiliteit:} Het project werkt op elke computer zonder
  pad-aanpassingen
\item
  \textbf{Collaboratie:} Anderen kunnen het project openen zonder
  pad-problemen
\item
  \textbf{Gescheiden omgevingen:} Elk project heeft zijn eigen workspace
  en history
\end{itemize}

\textbf{Stappen om een R Project aan te maken:}

\begin{enumerate}
\def\labelenumi{\arabic{enumi}.}
\item
  \textbf{Project Wizard openen:} Klik op het project-icoon rechtsboven
  in RStudio of ga naar File \textgreater{} New Project.
\item
  \textbf{Existing Directory kiezen:} Selecteer ``Existing Directory''
  omdat je al een thesis map hebt met de gewenste mapstructuur.
\end{enumerate}

\textbf{Figuur 1.4:} R Project Creation Wizard
\pandocbounded{\includegraphics[keepaspectratio,alt={R Project Wizard}]{screenshots and images/R project.png}}

\begin{enumerate}
\def\labelenumi{\arabic{enumi}.}
\setcounter{enumi}{2}
\tightlist
\item
  \textbf{Thesis map selecteren:} Browse naar je thesis map die de
  submappen (Data, Script, Literatures, Drafts, Output) bevat.
\end{enumerate}

\textbf{Figuur 1.5:} Select Project Directory
\pandocbounded{\includegraphics[keepaspectratio,alt={Select Directory}]{screenshots and images/Select_project folder.png}}

\begin{enumerate}
\def\labelenumi{\arabic{enumi}.}
\setcounter{enumi}{3}
\tightlist
\item
  \textbf{Project aanmaken:} Klik op ``Create Project'' om het R Project
  te voltooien.
\end{enumerate}

Na het aanmaken van het project zie je een \texttt{.Rproj} bestand in je
thesis map en de mapstructuur wordt automatisch zichtbaar in het Files
panel van RStudio.

\textbf{Figuur 1.6:} Project Structure after Creation
\pandocbounded{\includegraphics[keepaspectratio,alt={Project Created}]{screenshots and images/R projected created.png}}

\section{R Script Aanmaken en
Opslaan}\label{r-script-aanmaken-en-opslaan}

\textbf{Stap 4: Nieuw R Script aanmaken}

Na het opzetten van je project en mapstructuur, is het tijd om je eerste
R script aan te maken:

\begin{enumerate}
\def\labelenumi{\arabic{enumi}.}
\tightlist
\item
  \textbf{Nieuw script openen:} Ga naar File \textgreater{} New File
  \textgreater{} R Script (of gebruik Ctrl+Shift+N)
\end{enumerate}

\textbf{Figuur 1.7:} New R Script Creation
\pandocbounded{\includegraphics[keepaspectratio,alt={New R Script}]{screenshots and images/open_new_script.png}}

\begin{enumerate}
\def\labelenumi{\arabic{enumi}.}
\setcounter{enumi}{1}
\tightlist
\item
  \textbf{Script opslaan in Script map:} Sla het script ALTIJD op in de
  ``Script'' map van je project.
\end{enumerate}

\subsection{Naamgevingsconventie voor
Scripts}\label{naamgevingsconventie-voor-scripts}

\textbf{Belangrijk:} Gebruik ALTIJD de volgende naamgevingsconventie
voor je scripts:

\textbf{Format:} \texttt{DDMMYYYY\_beschrijving.R}

\textbf{Voorbeelden:} - \texttt{18112025\_hotspot\_analysis.R} -
\texttt{25112025\_descriptive\_statistics.R} -
\texttt{03122025\_regression\_analysis.R}

\textbf{Versiecontrole:} Wanneer je significante wijzigingen maakt aan
een bestaand script:

\begin{itemize}
\tightlist
\item
  \textbf{Bij werken op een andere datum:} Sla op met de nieuwe datum

  \begin{itemize}
  \tightlist
  \item
    Origineel: \texttt{18112025\_hotspot\_analysis.R}
  \item
    Nieuw bestand: \texttt{25112025\_hotspot\_analysis.R}
  \end{itemize}
\item
  \textbf{Bij meerdere versies op dezelfde dag:} Voeg versienummer toe

  \begin{itemize}
  \tightlist
  \item
    \texttt{25112025\_hotspot\_analysis\_V1.R}
  \item
    \texttt{25112025\_hotspot\_analysis\_V2.R}
  \end{itemize}
\end{itemize}

\textbf{Voordelen van deze methode:} - Je verliest nooit oude versies
van je werk - Duidelijk overzicht van wanneer wijzigingen zijn gemaakt -
Scripts staan automatisch op chronologische volgorde - Eenvoudig
teruggaan naar eerdere versies indien nodig - Supervisors kunnen de
evolutie van je werk volgen

\textbf{Tip:} Sorteer bestanden in Windows Explorer op ``Date modified''
om de meest recente versies bovenaan te zien.

\section{R Keyboard Shortcuts}\label{r-keyboard-shortcuts}

RStudio biedt een uitgebreide set sneltoetsen die je productiviteit
aanzienlijk kunnen verhogen. Deze shortcuts maken het mogelijk om snel
door je code te navigeren, code uit te voeren, en verschillende RStudio
functies te gebruiken zonder de muis.

\subsection{Toegang tot Keyboard
Shortcuts}\label{toegang-tot-keyboard-shortcuts}

\textbf{Complete shortcut lijst bekijken:} - \textbf{Methode 1:} Ga naar
Tools \textgreater{} Keyboard Shortcuts Help - \textbf{Methode 2:}
Gebruik de sneltoets \textbf{Alt + Shift + K}

\textbf{Figuur 1.9:} Accessing Keyboard Shortcuts Help
\pandocbounded{\includegraphics[keepaspectratio,alt={Keyboard Shortcuts Menu}]{screenshots and images/navigation_shorts.png}}

het opens window met alle keyboard shortcuts
\pandocbounded{\includegraphics[keepaspectratio,alt={Keyboard Shortcuts}]{screenshots and images/keyboard shortcuts.png}}

Deze opties openen een overzichtelijk venster met alle beschikbare
sneltoetsen, georganiseerd per categorie.

\section{Eerste Stappen in R}\label{eerste-stappen-in-r}

Nu je je project hebt opgezet en een R script hebt aangemaakt, kun je
beginnen met het leren van de basis van R programmeren.

\subsection{Commentaar in R}\label{commentaar-in-r}

\textbf{Belangrijk:} Alles wat je typt na een \texttt{\#} (hekje) wordt
beschouwd als commentaar en wordt NIET uitgevoerd door R.

\textbf{Waarom commentaar zo belangrijk is:} - \textbf{Voor jezelf:} Je
begrijpt je eigen code beter wanneer je er later naar terugkijkt -
\textbf{Voor anderen:} Collega's, supervisors, en reviewers kunnen je
code begrijpen - \textbf{Voor documentatie:} Het legt uit WAT je doet en
WAAROM je het doet

\begin{Shaded}
\begin{Highlighting}[]
\CommentTok{\# Dit is een commentaar {-} R voert deze regel niet uit}
\DecValTok{2} \SpecialCharTok{+} \DecValTok{3}  \CommentTok{\# Dit is ook een commentaar na de code}

\CommentTok{\# Uitleg van wat de volgende code doet}
\NormalTok{x }\OtherTok{\textless{}{-}} \DecValTok{10}  \CommentTok{\# Ken waarde 10 toe aan variabele x}
\NormalTok{y }\OtherTok{\textless{}{-}} \DecValTok{5}   \CommentTok{\# Ken waarde 5 toe aan variabele y}
\NormalTok{result }\OtherTok{\textless{}{-}}\NormalTok{ x }\SpecialCharTok{*}\NormalTok{ y  }\CommentTok{\# Bereken het product van x en y}
\end{Highlighting}
\end{Shaded}

\subsection{Code Secties Organiseren}\label{code-secties-organiseren}

\textbf{Tip:} Gebruik \texttt{\#\ sectienaam\ -\/-\/-\/-} (minimaal 3
streepjes) om secties te maken in je script. Dit maakt je code
overzichtelijker:

\begin{Shaded}
\begin{Highlighting}[]
\CommentTok{\# Data Cleaning {-}{-}{-}{-}}
\CommentTok{\# Hier doe je alle data cleaning activiteiten}

\CommentTok{\# Data Analysis {-}{-}{-}{-}  }
\CommentTok{\# Hier voer je je statistische analyses uit}

\CommentTok{\# Visualization {-}{-}{-}{-}}
\CommentTok{\# Hier maak je grafieken en plots}

\CommentTok{\# Tables {-}{-}{-}{-}}
\CommentTok{\# Hier creëer je tabellen voor je rapport}
\end{Highlighting}
\end{Shaded}

\subsubsection{Hoe Code Secties Werken}\label{hoe-code-secties-werken}

\textbf{Sectie Syntaxis:} - Begin altijd met \texttt{\#} gevolgd door
een spatie - Schrijf je sectienaam - Voeg minimaal 3 streepjes toe:
\texttt{-\/-\/-\/-} - Voorbeeld: \texttt{\#\ Data\ Cleaning\ -\/-\/-\/-}

\textbf{Navigeren met Table of Contents:} 1. \textbf{Open Table of
Contents:} Druk \textbf{Ctrl + Shift + O} 2. \textbf{Selecteer sectie:}
Klik op de gewenste sectie in de lijst 3. \textbf{Spring naar sectie:}
RStudio springt automatisch naar die regel 4. \textbf{Sluit overzicht:}
Druk \textbf{Escape} of klik ergens anders

\textbf{Figuur 1.10:} Code Sections Expanded with Table of Contents
\pandocbounded{\includegraphics[keepaspectratio,alt={Code Sections Expanded}]{screenshots and images/expanded and TOC.png}}

\textbf{Code Secties Inklappen/Uitklappen:} - \textbf{Alt + O:} Klap
alle code secties in (collapse all) - \textbf{Alt + Shift + O:} Klap
alle code secties uit (expand all) - \textbf{Alt + L:} Klap huidige
sectie in/uit (toggle current section)

\textbf{Figuur 1.11:} Code Sections Collapsed View
\pandocbounded{\includegraphics[keepaspectratio,alt={Code Sections Collapsed}]{screenshots and images/Collapased.png}}

Deze functies zijn bijzonder handig bij lange scripts waar je alleen
specifieke secties wilt zien en de rest wilt verbergen voor een cleaner
overzicht. Zoals je kunt zien in de afbeeldingen hierboven, geeft het
inklappen van secties een veel overzichtelijker beeld van je script
structuur.

\textbf{Voordelen van Code Secties:} - \textbf{Overzicht:} Zie in één
oogopslag de structuur van je script - \textbf{Snelle navigatie:} Spring
direct naar specifieke delen van je code - \textbf{Professionele
organisatie:} Maak je scripts leesbaar voor collega's - \textbf{Grote
scripts beheren:} Blijf overzicht houden in scripts van honderden regels

\textbf{Geavanceerde Sectie Tips:}

\begin{Shaded}
\begin{Highlighting}[]
\CommentTok{\# 1. Data Cleaning {-}{-}{-}{-}  }
\CommentTok{\# Clean en prep data voor analyse}

\CommentTok{\# 2. Data Analysis {-}{-}{-}{-}}
\CommentTok{\# Bekijk data distributie en patronen}

\CommentTok{\# 3. Visualization {-}{-}{-}{-}}
\CommentTok{\# Maak plots en grafieken}

\CommentTok{\# 4. Tables {-}{-}{-}{-}}
\CommentTok{\# Sla resultaten op voor rapport}
\end{Highlighting}
\end{Shaded}

Deze genummerde secties maken de workflow nog duidelijker en verschijnen
allemaal in de Table of Contents (Ctrl + Shift + O) voor gemakkelijke
navigatie door lange scripts.

\subsection{Code Uitvoeren}\label{code-uitvoeren}

Er zijn twee manieren om code uit te voeren in RStudio:

\begin{enumerate}
\def\labelenumi{\arabic{enumi}.}
\tightlist
\item
  \textbf{Run knop:} Klik op de ``Run'' knop bovenaan in het script
  venster
\item
  \textbf{Sneltoets:} Plaats je cursor op de regel en druk \textbf{Ctrl
  + Enter}
\end{enumerate}

\textbf{Figuur 1.8:} Run Button in RStudio
\pandocbounded{\includegraphics[keepaspectratio,alt={Run Button}]{screenshots and images/run_option.png}}

\subsection{Basis Rekenen in R}\label{basis-rekenen-in-r}

Laten we beginnen met enkele eenvoudige berekeningen. Typ de volgende
code in je script of direct in de console:

\begin{Shaded}
\begin{Highlighting}[]
\CommentTok{\# Eenvoudige berekening}
\DecValTok{2} \SpecialCharTok{+} \DecValTok{3}

\CommentTok{\# Andere rekenen operaties}
\DecValTok{10} \SpecialCharTok{{-}} \DecValTok{4}
\DecValTok{6} \SpecialCharTok{*} \DecValTok{7}
\DecValTok{20} \SpecialCharTok{/} \DecValTok{4}
\DecValTok{2}\SpecialCharTok{\^{}}\DecValTok{3}  \CommentTok{\# 2 tot de macht 3}
\end{Highlighting}
\end{Shaded}

\textbf{Probeer dit uit:} Typ \texttt{2\ +\ 3} in de console en druk op
\textbf{Enter}, of typ het in je script en druk \textbf{Ctrl + Enter}.
Je ziet de output \texttt{{[}1{]}\ 5} in de console.

\subsection{Variabelen Toewijzen}\label{variabelen-toewijzen}

In R gebruiken we \texttt{\textless{}-} (pijltje naar links) om waarden
toe te wijzen aan variabelen:

\begin{Shaded}
\begin{Highlighting}[]
\CommentTok{\# Variabelen aanmaken}
\NormalTok{x }\OtherTok{\textless{}{-}} \DecValTok{5}
\NormalTok{y }\OtherTok{\textless{}{-}} \DecValTok{7}

\CommentTok{\# Variabelen gebruiken in berekeningen}
\NormalTok{x }\SpecialCharTok{+}\NormalTok{ y      }\CommentTok{\# Optelling: 5 + 7 = 12}
\NormalTok{x }\SpecialCharTok{*}\NormalTok{ y      }\CommentTok{\# Vermenigvuldiging: 5 * 7 = 35}
\NormalTok{x }\SpecialCharTok{{-}}\NormalTok{ y      }\CommentTok{\# Aftrekking: 5 {-} 7 = {-}2}
\NormalTok{y }\SpecialCharTok{/}\NormalTok{ x      }\CommentTok{\# Deling: 7 / 5 = 1.4}
\NormalTok{x}\SpecialCharTok{\^{}}\DecValTok{2}        \CommentTok{\# x tot de macht 2: 5\^{}2 = 25}
\NormalTok{y}\SpecialCharTok{\^{}}\DecValTok{3}        \CommentTok{\# y tot de macht 3: 7\^{}3 = 343}
\end{Highlighting}
\end{Shaded}

\textbf{Let op:} Variabelnamen zijn hoofdlettergevoelig. \texttt{x} en
\texttt{X} zijn verschillende variabelen!

\textbf{Oefening:} 1. Maak een nieuw R script aan 2. Voeg commentaar toe
met je naam en datum 3. Probeer de bovenstaande voorbeelden uit 4.
Experimenteer met verschillende getallen en bewerkingen

\section{R Packages en Libraries}\label{r-packages-en-libraries}

R's echte kracht komt van de duizenden beschikbare packages (pakketten)
die extra functionaliteit bieden. Elke package is gespecialiseerd voor
specifieke taken, van data manipulatie tot geavanceerde statistische
analyses.

\subsection{Waarom Packages Gebruiken?}\label{waarom-packages-gebruiken}

\textbf{Base R vs.~Packages:} Base R bevat de basisfuncties, maar voor
efficiënte data-analyse hebben we gespecialiseerde packages nodig:

\begin{itemize}
\tightlist
\item
  \textbf{Data manipulatie:} \texttt{dplyr}, \texttt{tidyr}
\item
  \textbf{Data import/export:} \texttt{readr}, \texttt{readxl},
  \texttt{haven}
\item
  \textbf{Visualisatie:} \texttt{ggplot2}, \texttt{plotly}
\item
  \textbf{Statistische analyses:} \texttt{broom}, \texttt{car},
  \texttt{psych}, \texttt{MASS}
\item
  \textbf{Strings bewerken:} \texttt{stringr}
\item
  \textbf{Datum/tijd:} \texttt{lubridate}
\end{itemize}

\subsection{Packages Installeren}\label{packages-installeren}

\subsubsection{Methode 1: Via RStudio
Interface}\label{methode-1-via-rstudio-interface}

\begin{enumerate}
\def\labelenumi{\arabic{enumi}.}
\tightlist
\item
  \textbf{Packages Panel:} Ga naar het Packages panel (rechtsonder)
\item
  \textbf{Install Button:} Klik op ``Install''
\end{enumerate}

\textbf{Figuur 1.12:} Package Installation Navigation
\pandocbounded{\includegraphics[keepaspectratio,alt={Package Install Navigation}]{screenshots and images/lib instal nav.png}}

\begin{enumerate}
\def\labelenumi{\arabic{enumi}.}
\setcounter{enumi}{2}
\tightlist
\item
  \textbf{Package Name:} Typ de naam van het package
\item
  \textbf{Install:} Klik ``Install''
\end{enumerate}

\textbf{Figuur 1.13:} Package Installation Dialog
\pandocbounded{\includegraphics[keepaspectratio,alt={Package Install Dialog}]{screenshots and images/lib install diologbox.png}}

\subsubsection{Methode 2: Via Code
(Aanbevolen)}\label{methode-2-via-code-aanbevolen}

\begin{Shaded}
\begin{Highlighting}[]
\CommentTok{\# Package installeren (doe dit SLECHTS ÉÉN KEER)}
\FunctionTok{install.packages}\NormalTok{(}\StringTok{"dplyr"}\NormalTok{)}
\FunctionTok{install.packages}\NormalTok{(}\StringTok{"ggplot2"}\NormalTok{)}
\FunctionTok{install.packages}\NormalTok{(}\StringTok{"readr"}\NormalTok{)}

\CommentTok{\# Meerdere packages tegelijk installeren}
\FunctionTok{install.packages}\NormalTok{(}\FunctionTok{c}\NormalTok{(}\StringTok{"dplyr"}\NormalTok{, }\StringTok{"ggplot2"}\NormalTok{, }\StringTok{"readr"}\NormalTok{, }\StringTok{"tidyr"}\NormalTok{))}
\end{Highlighting}
\end{Shaded}

\textbf{Belangrijk:} Je hoeft een package maar \textbf{ÉÉN KEER} te
installeren op je computer. Daarna is het beschikbaar voor alle R
sessies.

\subsection{Packages Laden en
Gebruiken}\label{packages-laden-en-gebruiken}

\subsubsection{Methode 1: Library()
Functie}\label{methode-1-library-functie}

\begin{Shaded}
\begin{Highlighting}[]
\CommentTok{\# Package laden voor huidige sessie}
\FunctionTok{library}\NormalTok{(dplyr)}
\FunctionTok{library}\NormalTok{(ggplot2)}
\FunctionTok{library}\NormalTok{(readr)}

\CommentTok{\# Nu kun je alle functies gebruiken}
\FunctionTok{select}\NormalTok{(data, column1, column2)}
\FunctionTok{ggplot}\NormalTok{(data, }\FunctionTok{aes}\NormalTok{(x, y)) }\SpecialCharTok{+} \FunctionTok{geom\_point}\NormalTok{()}
\FunctionTok{read\_csv}\NormalTok{(}\StringTok{"data.csv"}\NormalTok{)}
\end{Highlighting}
\end{Shaded}

\subsubsection{Methode 2: Package::Function() Syntax (Aanbevolen voor
Leren)}\label{methode-2-packagefunction-syntax-aanbevolen-voor-leren}

\textbf{Voor beginners wordt deze methode sterk aanbevolen} omdat het
duidelijk maakt welke functie uit welk package komt:

\begin{Shaded}
\begin{Highlighting}[]
\CommentTok{\# Expliciet aangeven welk package je gebruikt}
\NormalTok{dplyr}\SpecialCharTok{::}\FunctionTok{select}\NormalTok{(data, column1, column2)}
\NormalTok{readr}\SpecialCharTok{::}\FunctionTok{read\_csv}\NormalTok{(}\StringTok{"data.csv"}\NormalTok{)}
\NormalTok{readxl}\SpecialCharTok{::}\FunctionTok{read\_excel}\NormalTok{(}\StringTok{"data.xlsx"}\NormalTok{)}
\NormalTok{ggplot2}\SpecialCharTok{::}\FunctionTok{ggplot}\NormalTok{(data, ggplot2}\SpecialCharTok{::}\FunctionTok{aes}\NormalTok{(x, y)) }\SpecialCharTok{+}\NormalTok{ ggplot2}\SpecialCharTok{::}\FunctionTok{geom\_point}\NormalTok{()}
\end{Highlighting}
\end{Shaded}

\subsection{Voordelen van Package::Function()
Syntax}\label{voordelen-van-packagefunction-syntax}

\textbf{Voor Leren:} - \textbf{Duidelijkheid:} Je ziet direct welk
package elke functie gebruikt - \textbf{Geen conflicten:} Vermijdt
problemen wanneer verschillende packages dezelfde functienaam hebben -
\textbf{Beter begrip:} Helpt bij het leren welke packages welke
functionaliteit bieden - \textbf{Reproduceerbarheid:} Scripts zijn
duidelijker voor anderen om te begrijpen

\textbf{Tip:} Naarmate je meer ervaring krijgt, kun je overstappen naar
het laden van packages met \texttt{library()}, maar voor leren is de
\texttt{package::function()} methode veel educatiever.

\end{document}
